%%%%%%%%%%%%%%%%%%%%%%%%%%%%%%%%%
%
% Lecture notes for Numerical Methods for Partial Differential Equations
%
% Chapter 2: Parabolic PDEs
%   Sections 5, 6, and 7
%
%%%%%%%%%%%%%%%%%%%%%%%%%%%%%%%%%

% !TeX root = NumPDE_Lecture_notes.tex

\subsection{Non-homogeneous heat equation with non-homogeneous boundary conditions}

\subsubsection{Non-homogeneous heat equation}

Consider first the non-homogeneous heat equation with homogeneous
boundary conditions, given by
\begin{eqnarray}
&&\frac{\pr u}{\pr t}(x,t) - K\frac{\pr^{2} u}{\pr x^{2}}(x,t)=f(x,t), \quad
0<x< L, \quad 0<t<T,   \label{d1} \\
&&u(0, t) = u(L, t)=0 \quad \hbox{for} \quad t>0,   \label{d2} \\
&&u(x, 0) = u_{0}(x).   \label{d3}
\end{eqnarray}
The generalisation of the finite-difference methods is straightforward.
 
\underline{Forward-difference method}:
\begin{equation}
\frac{w_{k,j+1}-w_{k,j}}{\tau}-K
\frac{w_{k+1, j}-2w_{kj}+w_{k-1,j}}{h^{2}}=f(x_k,t_j). \label{d10}
\end{equation}

\underline{Backward-difference method}:
\begin{equation}
\frac{w_{k,j}-w_{k,j-1}}{\tau}-K
\frac{w_{k+1, j}-2w_{kj}+w_{k-1,j}}{h^{2}}=f(x_k,t_j). \label{d11}
\end{equation}

\underline{Crank-Nicolson method}:
\begin{multline}
\frac{w_{k,j+1}-w_{k,j}}{\tau}-
K \frac{w_{k+1,j}-2w_{kj}+w_{k-1,j}+
w_{k+1,j+1}-2w_{k,j+1}+w_{k-1,j+1}}{2h^{2}}\\
=f(x_k,t_j+\tau/2). \label{d12}
\end{multline}
Note that the right hand side of Eq. (\ref{d12}) can be replaced by
$[f(x_k,t_j)+f(x_k,t_{j+1})]/2$ because
\[
f(x_k,t_j+\tau/2)-\frac{1}{2}[f(x_k,t_j)+f(x_k,t_{j+1})]=O(\tau^2).
\]
 
Do the above modifications affect the stability of the methods? The answer is NO, because
in spite of the fact that the difference equations are non-homogeneous, the corresponding equations for perturbation
$z_{kj}$ remain the same homogeneous equations as before.


  
 
\subsubsection{Non-homogeneous boundary conditions}

So far, we considered the homogeneous heat equation with homogeneous
(zero) boundary conditions \eqref{d2}. Now consider
the initial boundary value problem
\begin{eqnarray}
&&\frac{\pr u}{\pr t}(x,t) - K\frac{\pr^{2} u}{\pr x^{2}}(x,t)=f(x,t), \quad
0<x< L, \quad 0<t<T,   \label{d13} \\
&&u(0,t)=\mu_{1}(t), \quad u(L,t)=\mu_{2}(t) \quad \hbox{for} \quad t>0, \label{d14} \\
&&u(x, 0) = u_{0}(x),   \label{d15}
\end{eqnarray}
where $\mu_{1}(t)$ and $\mu_{2}(t)$ are some given functions.

The non-homogeneous boundary conditions (\ref{d14}) can be
dealt with in two ways. First, one can simply change
the boundary conditions for $w_{kj}$ by letting
\begin{equation}
w_{0,j}=\mu_{1}(t_j), \quad w_{N,j}=\mu_{2}(t_j). \label{d16}
\end{equation}
Note that for implicit schemes where we employ the double-sweep method, the latter should be modified
in order to allow non-zero boundary conditions.
 
Second, one can reduce the initial boundary value problem (\ref{d13})--(\ref{d15}) to an equivalent
problem with homogeneous boundary conditions. To do this, we choose any function
$g(x,t)$ satisfying the boundary conditions
\begin{equation}
g(0,t)=\mu_{1}(t), \quad g(L,t)=\mu_{2}(t). \label{d17}
\end{equation}
Now if $u(x,t)=v(x,t)+g(x,t)$ and $u(x,t)$ is the solution of
(\ref{d13})--(\ref{d15}), then $v(x,t)$ must satisfy the initial boundary value
problem
\begin{eqnarray}
&&\frac{\pr v}{\pr t}(x,t) - K\frac{\pr^{2} v}{\pr x^{2}}(x,t)=\tilde{f}(x,t), \quad
0<x< L, \quad 0<t<T,   \label{d18} \\
&&v(0,t)=0, \quad v(L,t)=0 \quad \hbox{for} \quad t>0, \label{d19} \\
&&v(x, 0) = v_{0}(x),   \label{d20}
\end{eqnarray}
where
\[
\tilde{f}(x,t)=f(x,t)-
\frac{\pr g}{\pr t}(x,t) + K\frac{\pr^{2} g}{\pr x^{2}}(x,t), \quad
v_{0}(x)=u_{0}(x)-g(x,0).
\]
For example, we can choose the function
\[
g(x,t)=\mu_{1}(t)+[\mu_{2}(t)-\mu_{1}(t)]\frac{x}{L}.
\]



%%%%%%%%%%%%%%%%%%%%%%%%%%%%%%%%%%%%%%%%%%%%%%%%%%%%%%%%%%%%%%%%%%%%%%%%%%%%%%%%%%%%%%%%%%%%%
%%%%%%%%%%%%%%%%%%%%%%%%%%%%%%%%%%%%%%%%%%%%%%%%%%%%%%%%%%%%%%%%%%%%%%%%%%%%%%%%%%%%%%%%%%%%%
%%%%%%%%%%%%%%%%%%%%%%%%%%%%%%%%%%%%%%%%%%%%%%%%%%%%%%%%%%%%%%%%%%%%%%%%%%%%%%%%%%%%%%%%%%%%%

\subsection{Boundary conditions of other types}

 
In all the problems we discussed so far, the
boundary conditions did not require approximations of any kind.
A different situation arises when we have the boundary conditions for the derivative
of $u$:
\begin{equation}
\frac{\pr u}{\pr x}(0,t)=\mu_{1}(t), \quad
\frac{\pr u}{\pr x}(L,t)=\mu_{2}(t). \label{d21}
\end{equation}
These are known as Dirichlet boundary conditions.
Evidently, we need to approximate these conditions.
 
Consider the non-homogeneous heat equation
\begin{equation}
\frac{\pr u}{\pr t} - K\frac{\pr^{2} u}{\pr x^{2}}=f(x,t), \quad
0<x< L, \quad 0<0<T,   \label{dd1}
\end{equation}
subject to the boundary conditions (\ref{d21}) and the initial condition
\begin{equation}
u(x, 0) = u_{0}(x).   \label{dd2}
\end{equation}
As in the case of non-homogeneous boundary conditions for $u$, the problem
(\ref{d21})--(\ref{dd2}) with non-homogeneous boundary conditions for $u_{x}$
can be reduced to
a problem with homogeneous boundary conditions for $u_{x}$. To do this, we write
$u(x,t)=v(x,t)+g(x,t)$ with any fixed function $g(x,t)$ satisfying conditions
(\ref{d21}). Then for $v(x,t)$, we obtain the problem
\begin{eqnarray}
&&\frac{\pr v}{\pr t} - K\frac{\pr^{2} v}{\pr x^{2}}=\tilde{f}(x,t), \quad
0<x< L, \quad 0<0<T,   \nonumber \\
&&\frac{\pr v}{\pr x}(0,t)=0, \quad
\frac{\pr v}{\pr x}(L,t)=0, \quad
v(x, 0) = v_{0}(x), \nonumber
\end{eqnarray}
where 
$$\tilde{f}(x,t)=f(x,t)-\frac{\pr g}{\pr t}(x,t)+K\frac{\pr^2 g}{\pr x^2}(x,t)$$ 
and $v_{0}(x)=u_{0}(x)-g(x,0)$.
[An example of function $g$: $g(x,t)=\mu_{1}(t)x+[\mu_{2}(t)-\mu_{1}(t)]x^2/(2L)$.]
Therefore, we will discuss here only the homogeneous conditions
\begin{equation}
\frac{\pr u}{\pr x}(0,t)=0, \quad
\frac{\pr u}{\pr x}(L,t)=0. \label{d22}
\end{equation}

In what follows we restrict our analysis to the boundary condition at $x=0$
(the other boundary condition can be treated similarly).
If we use the two-point forward-difference formula for the derivative
at $(x=0, t=t_{j})$, then
\[
\frac{w_{1,j}-w_{0,j}}{h}=0 .
\]
However, this formula for $u_{x}(0,t)$ has the truncation error $O(h)$,
while in all the methods that we have considered the truncation error
of the relevant difference equations was
at least $O(h^2)$. So, the use of this formula would increase the local
truncation error of these methods.

  
 
Suppose that we use the forward-difference method
to approximate the heat equation (\ref{dd1}) at the interior grid points.
How to approximate the boundary conditions (\ref{d22})
with truncation error $O(h^2)$ for this method?
We will consider two ways of doing this.
In the first one, we add a `false' boundary at
$x=x_{-1}=x_{0}-h$ and assume that the forward-difference scheme
approximates the heat equation at points $(x_{0},t_{j})$ ($j=1,2,\dots$). Then,
we have
\begin{equation}
\frac{w_{0,j+1}-w_{0,j}}{\tau}-K
\frac{w_{1, j}-2w_{0j}+w_{-1,j}}{h^{2}}=f(0,t_j). \label{d23}
\end{equation}
Approximating $u_{x}({0},t_{j})$ by the central difference formula
(whose truncation error is $O(h^2)$), we find that
\begin{equation}
\frac{w_{1,j}-w_{-1,j}}{2h}=0. \label{d24}
\end{equation}
Eliminating $w_{-1,j}$ from Eqs. (\ref{d23}) and (\ref{d24}), we obtain
\begin{equation}
w_{0,j+1}=(1-2\gamma)w_{0,j}+2\gamma w_{1,j}+\tau f(0,t_j). \label{d25}
\end{equation}
This is an explicit formula that relates the boundary values at the time levels
$t_{j}$ and $t_{j+1}$.

  
 
Similarly, for the boundary condition at $x=L$, one can obtain the formula
\begin{equation}
w_{N,j+1}=(1-2\gamma)w_{N,j}+2\gamma w_{N-1,j}+\tau f(L,t_j). \label{d26}
\end{equation}

  
 
In another (more general) technique which leads to Eqs. (\ref{d25}) and
(\ref{d26}), we expand $u(x_{1},t_{j})$ in Taylor's series at point
$(x_{0},t_{j})$:
\[
u(x_{1},t_{j})=u(x_{0},t_{j})+h u_{x}(x_0,t_j)+
\frac{h^2}{2}u_{xx}(x_0,t_j)+O(h^3).
\]
Taking account of the boundary condition and the fact that $u(x,t)$ is
the solution of the heat equation, we find that
\begin{eqnarray}
u(x_{1},t_{j})=u(x_{0},t_{j})&+&
\frac{h^2}{2K}\left(\frac{\pr u}{\pr t}(x_0,t_j)-
f(x_0,t_j)\right)+O(h^3) \nonumber \\
= u(x_{0},t_{j})&+&
\frac{h^2}{2K}\left(\frac{u(x_0,t_{j+1})-u(x_0,t_{j})}{\tau}-
f(x_0,t_j)\right)\nonumber\\
&+&O(\tau h^2)+O(h^3). \nonumber
\end{eqnarray}
The last equation leads to the difference equation (\ref{d25}).
 

 
More general boundary conditions (Robin boundary conditions)
\begin{equation}
\frac{\pr u}{\pr x}(0,t)+c_{1}(t)u(0,t)=\mu_{1}(t), \quad
\frac{\pr u}{\pr x}(L,t)+c_{2}(t)u(L,t)=\mu_{2}(t), \label{d27}
\end{equation}
where $c_{1}(t)$ and $c_{2}(t)$ are some given functions, can be treated
similarly.


%%%%%%%%%%%%%%%%%%%%%%%%%%%%%%%%%%%%%%%%%%%%%%%%%%%%%%%%%%%%%%%%%%%%%%%%%%%%%%%%%%%%%%%
\subsection{Variable coefficients}.

Consider the parabolic equation
\begin{equation}
\frac{\pr u}{\pr t} =a(x,t)\frac{\pr^{2} u}{\pr x^{2}}+
b(x,t)\frac{\pr u}{\pr x}+c(x,t)u+d(x,t),  \quad 0<x< L, \quad 0<t<T, \label{e1}
\end{equation}
subject to the initial and boundary conditions
\begin{eqnarray}
&&u(x, 0) = u_{0}(x), \label{e2} \\
&&u(0,t)=0, \quad u(L,t)=0.   \label{e3}
\end{eqnarray}
Here we assume that
\begin{equation}
a(x,t)> 0 \quad {\rm for} \quad 0\leq x\leq L, \ \ 0<t<T. \label{e4}
\end{equation}
Most of the previously discussed methods can be generalized to the case of
the initial boundary value problem (\ref{e1})--(\ref{e3}).  How to do this?
 
Employing the forward difference formula for $u_t$ and
the central difference formulae for $u_x$ and $u_{xx}$, we obtain
the following finite-difference approximation for Eq. (\ref{e1}):
\begin{equation}
\frac{w_{k,j+1}-w_{k,j}}{\tau} =a_{kj}\frac{w_{k+1,j}-2w_{k,j}+w_{k-1,j}}{h^{2}}+
b_{kj}\frac{w_{k+1,j}-w_{k-1,j}}{2h}+c_{kj}w_{k,j}+d_{kj},   \label{e5}
\end{equation}
for $k=1, \dots,N-1$ and $j=0, 1,\dots,M-1$. In Eq. (\ref{e5}),
$a_{kj}=a(x_{k},t_{j})$, $b_{kj}=b(x_{k},t_{j})$, etc. Since
the forward difference formula
for $u_t$ has a truncation error that is $O(\tau)$, and
the errors of the central difference formulae for $u_x$ and $u_{xx}$
are $O(h^2)$, the local truncation error of the difference equation (\ref{e5}) is
\[
\tau_{kj}=O(\tau+h^2).
\]
This, together with obvious boundary conditions, gives us the explicit method
for solving (\ref{e1})--(\ref{e3}) which is similar to the explicit
forward-difference method for the heat equation that we discussed earlier.
Similarly, if we use the backward difference formula for $u_t$, we obtain
the difference equation
\begin{equation}
\frac{w_{k,j}-w_{k,j-1}}{\tau} =a_{kj}\frac{w_{k+1,j}-2w_{k,j}+w_{k-1,j}}{h^{2}}+
b_{kj}\frac{w_{k+1,j}-w_{k-1,j}}{2h}+c_{kj}w_{k,j}+d_{kj},   \label{e6}
\end{equation}
whose local truncation error is $O(\tau+h^2)$ and which is similar to the implicit
backward-difference method for the heat equation.
 
Crank-Nicolson's technique applied to (\ref{e1}) yields:
\begin{eqnarray}
\frac{w_{k,j+1}-w_{k,j}}{\tau} &=& \frac{a_{k,j+1/2}}{2h^{2}} \,
\delta_{x}^2 \, \left(w_{k,j}+w_{k,j+1}\right)
+\frac{b_{k,j+1/2}}{4h} \, \delta_{x} \,  \left(w_{k,j}+w_{k,j+1}\right)\nonumber \\
&&+\frac{c_{k,j+1/2}}{2} \, \left(w_{k,j}+w_{k,j+1}\right)+d_{k,j+1/2}, \label{cnvc}
\end{eqnarray}
where
\[
a_{k,j+1/2}=\frac{a(x_{k},t_{j})+a(x_{k},t_{j+1})}{2}, \quad
b_{k,j+1/2}=\frac{b(x_{k},t_{j})+b(x_{k},t_{j+1})}{2}, \ \ {\rm etc,}
\]
and we have denoted the central-difference operator for the first derivative by 
$\delta_{x}$.
The local truncation error of the approximation (\ref{cnvc}) can be shown to be
$O(\tau^2+h^2)$.
 
%What about the stability of these methods?
%If the coefficients $a$, $b$, $c$ and $d$ do not
%involve $t$, the stability can be investigated
%\footnote{See W. F. Ames, Numerical methods for partial differential equation,
%Academic Press, 1977.}. In particular, it can be shown that the explicit method
%(\ref{e5}) is stable provided that
%\[
%\frac{\tau}{h^2}<\frac{1}{2a(x,t)} \quad {\rm for} \quad 0\leq x\leq L, \ \ t>0.
%\]

\begin{example}
Consider the two-dimensional heat equation
\begin{equation}
\frac{\partial u}{\partial t}= \frac{\partial^{2}u}{\partial
x^{2}}+ \frac{\partial^{2}u}{\partial y^{2}} \quad \textrm{for} \ \ 0<t<T \label{e7}
\end{equation}
in the circular domain ${\cal D}$:
\[
{\cal D}=\{(x,y) \vert \sqrt{x^2+y^2}< R \},
\]
subject to the boundary condition
\begin{equation}
u(x,y,t)=\mu(x,y,t) \quad \hbox{for} \quad \sqrt{x^2+y^2}= R,  \label{e8}
\end{equation}
and the initial condition
\begin{equation}
u(x,y,0)=u_{0}(x,y) \quad \hbox{for} \quad (x,y)\in{\cal D}.  \label{e9}
\end{equation}
In polar coordinates $(r, \theta)$ [such that $x=r\cos\theta$, $y=r\sin\theta$],
the above initial boundary value problem takes the form
\begin{eqnarray}
&&\frac{\partial u}{\partial t}=\frac{\partial^{2}u}{\partial r^{2}}
+ \frac{1}{r}\frac{\partial u}{\partial r}+
\frac{1}{r^2}\frac{\partial^{2}u}{\partial \theta^{2}}
\quad {\rm in} \quad {\cal D},  \nonumber \\
&&u(r,\theta,t)\vert_{r=R}=\mu(\theta,t), \quad
u(r,\theta, 0)=u_{0}(r,\theta).  \label{e10}
\end{eqnarray}
If the initial and boundary conditions do not involve
$\theta$, i.e. $u_{0}=u_{0}(r)$ and $\mu=\mu(t)$, then
the solutions of problem (\ref{e10}) are rotationally symmetric, i.e.
independent of $\theta$, for all $t>0$. In this case, $(\ref{e10})$
simplifies to
\begin{eqnarray}
&&\frac{\partial u}{\partial t}=\frac{\partial^{2}u}{\partial r^{2}}
+ \frac{1}{r}\frac{\partial u}{\partial r}
\quad {\rm for} \quad 0 < r < R,  \nonumber \\
&&u(r,t)\vert_{r=R}=\mu(t), \quad
u(r, 0)=u_{0}(r).  \label{e11}
\end{eqnarray}
There are two apparent difficulties with problem (\ref{e11}): (i) we have only one boundary condition
and (ii) the term $(1/r)(\pr u/\pr r)$ is singular at $r = 0$.
The first difficulty can be eliminated by using the following condition
\begin{equation}
u_r(0,t)=0.  \label{e12}
\end{equation}
This condition holds for any sufficiently smooth solution of (\ref{e11}). To show this, we assume that $u$ is twice continuously differentiable with respect to both $t$
and $r$ in $D$ and integrate Eq. (\ref{e11}) in $r$ from $0$ to $\eps>0$ with weight $r$:
\[
\int\limits_{0}^{\eps}u_{t}(r,t) r \, dr = \int\limits_{0}^{\eps}\frac{1}{r}\, \frac{\pr}{\pr r}\Bigl(r \, u_{r}(r,t)\Bigr) r \, dr
=r \, u_{r}(r,t)\Bigm\vert_{0}^{\eps}= \eps \, u_{r}(\eps,t).
\]
Hence,
\begin{equation}
u_{r}(\eps,t)=\frac{1}{\eps} \, \int\limits_{0}^{\eps}u_{t}(r,t) r \, dr .  \label{e12b}
\end{equation}
For small $\eps$,
\[\begin{split}
\int\limits_{0}^{\eps}u_{t}(r,t) r \, dr &= \int\limits_{0}^{\eps}(u_{t}(0,t)+O(r)) r \, dr=
u_{t}(0,t)\int\limits_{0}^{\eps} (r + O(r^2)) \, dr \\
&=u_{t}(0,t) \, \frac{\eps^2}{2} + O(\eps^3)
\end{split}\]
Passing to the limit as $\eps\to 0$ in (\ref{e12b}) leads to condition (\ref{e12}):
\[
u_{r}(0,t)=\lim_{\eps\to 0} u_{r}(\eps,t)=\lim_{\eps\to 0} \frac{1}{\eps} \, \int\limits_{0}^{\eps}u_{t}(r,t) r \, dr
=\lim_{\eps\to 0} \frac{1}{\eps} \, \left(u_{t}(0,t) \, \frac{\eps^2}{2} + O(\eps^3) \right)=0.
\]
If we try to approximate the boundary condition (\ref{e12}) by introducing a false boundary at $r=r_{-1}=-h$ in the same way as before,
we will need to use a finite-difference scheme at points $(r_{0},t_j)=(r_{0},t_j)$ and therefore to approximate the term $r^{-1}u_{r}$ at $r=0$.
However, we do not know $\lim_{r\to 0}r^{-1}u_{r}$ and, therefore, cannot approximate it. Nevertheless we can avoid the singularity at $r=0$, if we choose
the grid points in such a way that $r_{0}=h/2$ and $r_{k}=r_0+k h$ for $k=1,\dots,N$ with $h=2L/(2N+1)$. Then we introduce the false boundary at $r_{-1}=-h/2$ and approximate
(\ref{e12}) by the central difference formula
\begin{equation}
\frac{w_{0,j}-w_{-1,j}}{h}=0  \label{e12a}
\end{equation}
whose truncation error is $O(h^2)$.

When the Crank-Nicolson concept is applied to Eq. (\ref{e11}),
we obtain the difference equation for the interior grid points $(r_k,t_j)$:
\begin{equation}
\frac{w_{k,j+1}-w_{k,j}}{\tau}-
\frac{1}{2h^2}\left(\delta_{r}^2+\frac{h}{2r_{k}}\delta_{r}\right)
\left(w_{k,j+1}+w_{kj}\right)=0  \label{e13}
\end{equation}
for $k=1,\dots,N-1)$ and $j=0,\dots,M-1$.
Here $\delta^2_{r}$ and $\delta_{r}$ are finite-difference operators, defined
by
\begin{equation}
\delta^2_{r}w_{kj}=w_{k+1,j}-2w_{kj}+w_{k-1,j} \quad {\rm and} \quad
\delta_{r}w_{kj}=w_{k+1,j}-w_{k-1,j}.  \label{e14}
\end{equation}
For grid points $(r_0,t_j)$, we use the same difference equations with $w_{-1,j}$ eliminated using
(\ref{e12a}), i.e.
\begin{equation}
\frac{w_{0,j+1}-w_{0,j}}{\tau}-
\frac{1}{2h^2}\left(1+\frac{h}{2r_{0}}\right)
\left(w_{1,j+1}-w_{0,j+1}+w_{1,j}-w_{0,j}\right)=0 
\end{equation}
for $j=0, 1, \dots, M-1$.

\end{example}
 
%\begin{example} One of the most well-known equation in mathematical finance is the
%the Black-Scholes equation:
%\begin{equation}
%    \frac{\partial V}{\partial t}+\frac{1}{2}\sigma^{2}S^{2}\frac{\partial^{2} V}{\partial S^{2}}+rS\frac{\partial V}{\partial S}
%    -rV=0. \label{BS1}
%\end{equation}
%Here $V$ is the value of the option, $\sigma$ and $r$ are constants.
%$V$ depends on the current value $S$ of the underlying asset and on time $t$.
%
%When this equation describes European call option, it is solved subject
%to the following conditions:
%\begin{eqnarray}
%    &&V(0,t)=0 \ \ \text{for} \ \ t<T, \label{BS2} \\
%    &&V(S,t)\rightarrow S-Ee^{-r(T-t)}\text{ as }S\rightarrow\infty\, , \label{BS3} \\
%    &&V(S,T)=\max\{S-E,0\}. \label{BS4}
%\end{eqnarray}
%In Eq. (\ref{BS4}), $E$ is a given positive constant. The problem is to solve
%Eq. (\ref{BS1}) backwards in time, i.e. for $t<T$. In principle, Eq. (\ref{BS1})
%can be transformed to the heat equation on the whole lines and then solved exactly.
%Our task is to obtain a numerical solution.
%
%It is convenient to introduce a new independent variable $\tilde{t}=T-t$
%and a new dependent variable $\tilde{V}(S,\tilde{t})=V(S,t)-S+Ee^{-r(T-t)}(1-e^{-S})$. Then problem (\ref{BS1})--(\ref{BS4})
%takes the form
%\begin{eqnarray}
%    &&\frac{\partial \tilde{V}}{\partial \tilde{t}}=\frac{1}{2}\sigma^{2}S^{2}
%    \frac{\partial^{2} \tilde{V}}{\partial S^{2}}+rS\frac{\partial \tilde{V}}{\partial S}
%    -r\tilde{V}+ f(S, \tilde{t}), \label{BS5} \\
%    &&\tilde{V}(0,\tilde{t})=0 \ \ \text{for} \ \ t<T, \label{BS6} \\
%    &&\tilde{V}(S,\tilde{t})\rightarrow 0\text{ as }S\rightarrow\infty\, , \label{BS7} \\
%    &&\tilde{V}(S,0)=\max\{0,E-S\}-E e^{-S}. \label{BS8}
%\end{eqnarray}
%Here $f(S, \tilde{t})=E \left(\frac{1}{2}\sigma^2 S^2 - r S\right)\, e^{-r\tilde{t}-S}$.
%Problem (\ref{BS5})--(\ref{BS8}) is an initial boundary value problem for a parabolic
%equation with variable coefficients.
%
%To solve the problem, we choose sufficiently large $S_{0}$ and define the grid points:
%\[
%(S_k,\tilde{t}_j)=(hk, \tau j) \ \ \text{for} \ \ k=0,\dots,N \ \ \text{and} \ \ j=0, 1, \dots, M
%\]
%where $h=S_{0}/N$ and $\tau=T/M$ is the step length in time $\tilde{t}$.
%
%To make sure  that out scheme is stable, we employ the backward-difference formula
%for $\partial \tilde{V}/\partial \tilde{t}$ and central difference formulae for
%$\partial^{2} \tilde{V}/\partial S^{2}$ and $\partial \tilde{V}/\partial S$.
%As  a result, we have
%\begin{eqnarray}
%    &&\frac{w_{k,j}-w_{k,j-1}}{\tau}=\frac{1}{2}\sigma^{2}S_k^{2}
%    \frac{w_{k+1,j}-2w_{k,j}+w_{k-1,j}}{h^2}\nonumber\\
%    &&\qquad\qquad\qquad\qquad+rS_k\frac{w_{k+1,j}-w_{k-1,j}}{2h}
%    -r w_{k,j}+f_{k,j}, \label{BS9} \\
%    &&w_{0,j}=0, \quad w_{N,j}=0 , \label{BS11} \\
%    &&w_{k,0}=\max\{0,E-S_k\}-E e^{-S_k}. \label{BS12}
%\end{eqnarray}
%These equations can be solved using the standard double-sweep method.
%\end{example}
